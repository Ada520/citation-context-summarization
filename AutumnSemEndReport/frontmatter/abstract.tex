% These \phantomsection are to ensure that the hyperref package hyperlinks to the correct page in the electronic pdf. If you turn hyperref off they don't do anything so they can just stay here.
\phantomsection\addcontentsline{toc}{chapter}{Abstract}
\chapter*{Abstract} % Starred chapter=chapter with no number.
Quickly moving to a new area of research is painful for researchers due to the vast amount of scientific literature in each field of study. One possible way to overcome this problem is to summarize a scientific
topic. Our goal is to effectively solve this problem by using bibliometric text mining and summarization techniques to generate summaries of scientific literature. It is a proven fact that we can use citations to produce automatically generated, readily consumable, technical extractive summaries.We generate extractive summaries of a set of Question Answering (QA) and
Dependency Parsing (DP) papers, their abstracts, and their citation sentences and show
that citations have unique information amenable to creating a summary. This work is built upon C-LexRank, a model for summarizing single scientific articles based on citations, which employs community detection and extracts salient information-rich sentences. The work focuses on improving the efficiency of the state-of-the-art C-LexRank algorithm by using new similarity measures, fine-tuning various modules of the algorithm for a better performance than the baseline. We introduce a host of new features namely Unigram similarity, Bigram similarity, Number of citations, Unigram similarity of Part-of-speech tags, Bigram Similarity of Part-of-speech tags, Bibilographic coupling metrics, Co-citation metrics, Title similiarity, Author Similarity \& Temporal similiarity. We introduce a concept of thresholding the edges of similarity graph in the community detection of C-LexRank to improve the performance.